\begin{document}
\documentclass{article}

\begin{document}

There are various types of Linux Distros that you can try:

\begin{itemize}

\item [\ding{71}] We say that data follows a cycle because once used, it can generate new data, sometimes more precise.

\item [\ding{71}] En 1997 une intelligence artificielle a battu le meilleur joueur d'échecs. 
Elle s'est notamment servie des données récoltées de précédentes parties pour faire ses choix.


Pour adapter ses messages à ses différents clients, une marque peut s’appuyer sur les données qu’elle récolte au quotidien pour: Créer des segments de client, Catégoriser des clients aux comportements similaires, Adresser des messages personnalisés


Il existe d'autres types d'analyse : l'analyse prescriptive, l'analyse descriptive et l'analyse de diagnostic.

\section{Data portability} = protect users from having their data stored in "silos" or "walled gardens" that are incompatible with one another, i.e. closed platforms.


\blindtext

\section{GDPR}
Adopted by the European Commission in April 2016, the GDPR enters into force on 25 May 2018 for all member states of the European Union. It is intended to unify the European legal framework for data protection.

Data circulating internally is is exposed to risks of interception. For this reason, emails sent between employees should be encrypted encrypted.
Encryption of communications is not a new phenomenon. The Enigma machine, developed in 1918 and used during the Second World War, was an early example of a system for automatically encrypting and decrypting strategic information.



It is essential to protect data circulating internally from external threats, including emails. As they pass onto the company’s servers, they are exposed to interception attempts. For their protection, emails must be end-to-end encrypted. In this way, only the sender and the recipient can read them.



\blindtext


\end{itemize}

\end{document}
