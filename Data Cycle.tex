\begin{document}
\documentclass{article}

\begin{document}

There are various types of Linux Distros that you can try:

\begin{itemize}

\item [\ding{71}] We say that data follows a cycle because once used, it can generate new data, sometimes more precise.

\item [\ding{71}] En 1997 une intelligence artificielle a battu le meilleur joueur d'échecs. 
Elle s'est notamment servie des données récoltées de précédentes parties pour faire ses choix.


Pour adapter ses messages à ses différents clients, une marque peut s’appuyer sur les données qu’elle récolte au quotidien pour: Créer des segments de client, Catégoriser des clients aux comportements similaires, Adresser des messages personnalisés


Il existe d'autres types d'analyse : l'analyse prescriptive, l'analyse descriptive et l'analyse de diagnostic.

\section{Data portability} = protect users from having their data stored in "silos" or "walled gardens" that are incompatible with one another, i.e. closed platforms.


\blindtext

\section{GDPR}
Adopted by the European Commission in April 2016, the GDPR enters into force on 25 May 2018 for all member states of the European Union. It is intended to unify the European legal framework for data protection.

Data circulating internally is is exposed to risks of interception. For this reason, emails sent between employees should be encrypted encrypted.
Encryption of communications is not a new phenomenon. The Enigma machine, developed in 1918 and used during the Second World War, was an early example of a system for automatically encrypting and decrypting strategic information.


Not all machines or all users have the same security requirements. A printer connected to the internal network is much less exposed than a computer connected to the internet. In order to protect them, machines and users should be grouped into systems sharing the same security requirements.

It is essential to protect data circulating internally from external threats, including emails. As they pass onto the company’s servers, they are exposed to interception attempts. For their protection, emails must be end-to-end encrypted. In this way, only the sender and the recipient can read them.

Setting up and managing a minimum security level are the responsibility of the ISD. Consequently, small structures which do not have the ability to employ an ISD and which have not defined a minimum security base are highly exposed to the risk of computer attacks.

Algorithms will first "test" words contained in dictionaries and not including any so-called complex characters, such as accents and numbers, and special characters such as the following punctuation symbols: , : - _ / ! ? .

\section{Cognitive security}

Cognitive security tools will read and learn from tens of thousands of available sources of information. This extensive knowledge will uncover new insights and patterns, and it will allow to anticipate, isolate and minimize attacks as they happen.
Because cybercrime is nowadays organized and heavily funded, many criminals are turning to advanced technologies such as artificial intelligence. We might be moving towards a future where AI security tools respond to AI-powered threats.
\\

Irregular activity often indicates that an account is under attack. 
Suspicious behavior can consist of \textbf{late night log-ins and an unusual amount of data being downloaded}, for example. 
  
A user behavior analytics engine is a tool that monitors account activity and identifies those signs.
\\

Today’s cybercrime is far from lone hackers of the past. Now, large organized crime rings function like start-ups and often employ highly trained developers who are constantly innovating in the field of online attacks.
\blindtext



\section{App wrapping}

An important step after the development of an app is app wrapping. This procedure adds multiple security layers of protection, without making any changes to the app itself. It allows the app manager to set specific policy elements that will contribute to minimizing the risks of data being stolen. For example, it can prevent users from cutting and pasting sensitive information when they want to log in.

\blindtext




To prevent data leakage, the loss or unauthorized transfer of any sensitive information, the maker of an app must first ensure that all important data is encrypted. Beyond this encryption, there are a number of security measures that will help to prevent the app from being misused.



On average, Americans checked their phones 262 times per day in 2021—that's once every 5.5 minutes! In the land of unlimited data plans and stupefying smartphones, we surveyed Americans 18 and older about their phone-related behaviors to see how far we've fallen into our screens.24 janv. 2022

 \textbf{Phishing} is the attempt to gain sensitive information by masquerading as a trusted source. It can, for example, take the form of a fraudulent email coming from a friend.


 \textbf{Privileged Identity Management system}
 Privileged identity management (PIM) gives users the ability to control, manage, and monitor the access privileges that people have to crucial resources within an organization. These may include important files, user accounts, documentation, and even application code and infrastructural elements such as databases and security systems.



In the case of non-compliance with the GDPR, by default the DPO will not be held personally responsible. Complying with personal data protection is the responsibility of the data controller, and in the case of shared processing, the data processor.

It is the data controller’s responsibility to comply with the application of the GDPR’s six main principles and with the individual’s eight rights.
In the case of a possible personal data breach, the data controller must inform the supervisory authority. They must do it as quickly as possible, no longer than 72 hours after having learned about it.



 \textbf{ }
The implementation date of the General Data Protection Regulation raises a lot of skepticism and misconceptions.

A firm worldwide deadline, The beginning of possible prosecution

Data protection from the start and by default, Shared responsibility among the players, The ability to always demonstrate compliance

Lawfulness, fairness and transparency
Purpose limitation
Data minimisation
Accuracy
Storage limitation
Integrity and confidentiality (security)
Accountability


The action plan must be structured around four dimensions: the clients, the partners, the processors and the employees of the company or organisation.

The GDPR awareness meeting helps managers analyse the company’s roles and challenges through practical case studies of personal data processing.

The data protection officer, The head of the project on compliance

In early 2018, the GDPR authorised personal data to circulate freely and without restriction between 31 countries. This group includes the 28 member states of the EU (along with the United Kingdom, as long as it remains an official member), as well as Iceland, Norway and Liechtenstein.


Is the data being sent between multiple countries?, Is it processing personal data?, Was the data provided by the individuals concerned?, For which activity is the data being processed?, Is the data being sent to a third party?


The company must answer five key questions, which will enable it to establish an initial analysis. This analysis is the starting point for the action plan. The company must first ask itself if it is processing personal data. If so, it must be determined whether the data has been provided by the individuals concerned, whether it has been transmitted to a third party and whether it is being sent between multiple countries. The company must also define the activities for which data is processed.


Document data processing, Set up data processing audits
Once the company has deployed its action plan, it must constantly document and update its personal data processing. Close supervision of its processing must also be set up. GDPR compliance is a constant cycle, which will continue on with every change in the processes or ecosystem of the organisation.

Once the company has deployed its action plan, it must constantly document and update its personal data processing. Close supervision of its processing must also be set up. GDPR compliance is a constant cycle, which will continue on with every change in the processes or ecosystem of the organisation.




t. Il existe 5 typologies de données :
Données identitaires :
• Les données identitaires présentent 2 notions : les informations liées à un compte
que possède un individu, un identifiant de compte par exemple / une information
externe (non générée par l’entreprise) qui permet d’identifier un individu de manière
unique, un email, un numéro de sécurité sociale
Données démographiques :
• viennent enrichir les données identitaires et nous donnent des indications sur les
préférences, les intérêts d’un individu : nom, numéro de téléphone, composition de
la famille, projets, revenus etc. ex. : déménagement peut intéresser une entreprise
qui vend des contrats d’électricité
• les données démographiques ont longtemps été utilisées seules pour segmenter les
clients (c’est à dire créer des sous-ensemble d’individus selon des critères qui les
unissent). Cela revient par exemple à dire que les personnes d’un même âge seront
intéressées par le même produit. Pour avoir une vision plus globale et réaliste, il faut
compléter avec les données transactionnelles, d’interaction et comportementales.
Données transactionnelles :
• quant à elles sont les informations d’achat, de produits ou de services réalisées par
un individu. On y retrouve également les devis ou les paniers constitués par un
visiteur.
Données d’interactions :
• regroupent toutes les interactions qu’un individu a avec l’entreprise, sur l’ensemble
des canaux. On y retrouve les échange textuels : email, sms, chat, sur les réseaux
sociaux / les échanges vocaux : lorsqu’un client va appeler le service client par
exemple / les échanges en personne : dans le cas d’une visite en magasin par
exemple
• combinées avec les bons outils, la bonne analyse et une réelle stratégie omnicanale,
ces données peuvent permettre l'attribution d’indicateurs, d’alertes potentielles, et
ainsi de mieux répondre aux problématiques des clients.
Données comportementales :
• on retrouve dans les données comportementales les informations liées à l’activité
d’un individu sur un site internet. Ce sont par exemple des habitudes de navigation,
des clics sur une certaine rubrique, l’utilisation d’une application ou d’un site mobile.
Mais le comportement ne se limite pas à l’activité en ligne, les réponses
émotionnelles, comportements en magasin, expressions faciales sont autant de
données à analyser dans le monde physique.
\section{Bibliography (to change)}
https://www.ibm.com/events/think/
\blindtext
\end{itemize}

\end{document}
