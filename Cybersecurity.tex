\documentclass[lualatex,10pt,unknownkeysallowed]{beamer}
\usepackage{style/arpafvg}
\usepackage{hyperref}
\usepackage{siunitx}
\usepackage{color, colortbl}
\definecolor{Gray}{gray}{0.9}
\usepackage{lipsum}

%% titolo ecc
\title{\textit{Cyber security}\LaTeX}
\subtitle{conforme ai formati ufficiali di ARPA-FVG}
\date[\today]{Palmanova, \today}
\author[G.Bonafè]{Giovanni Bonafè}
\institute{Centro Regionale di Modellistica Ambientale, ARPA-FVG}

\begin{document}

\titlepage

\begin{frame}
\frametitle{premessa} 
Se non volete rinunciare a beamer+LaTeX per le presentazioni, nel rispetto delle prescrizioni dell'Area Comunicazione, potete provare a usare questo \textit{template}.
\end{frame}

\begin{frame}
\frametitle{sorgenti}
Per scaricare il template\footnote{non l'ho provato su FENICE, temo che per usare il font ufficiale \textbf{Decima WE} usando questo template sia necessario avere \textbf{lualatex} o \textbf{xelatex} } (inclusi i logo) e modificarlo a piacimento: \par
\texttt{git clone https://git.overleaf.com/5bb32ba687ec791b18393ec3}
\vfill
Oppure lo trovate qui:\par
\url{https://www.overleaf.com/read/gvcjvvcxkhfx}
\end{frame}

\begin{frame}
\frametitle{una slide con un testo lungo e un titolo su due righe}
\blindtext
\end{frame}

\section{una nuova sezione}

\begin{frame}
\frametitle{una slide con elenchi e formule}
\begin{itemize}
\item una cosa
\begin{equation}
F(x,y)=0 ~~\mbox{and}~~
\left| \begin{array}{ccc}
  F''_{xx} & F''_{xy} &  F'_x \\
  F''_{yx} & F''_{yy} &  F'_y \\
  F'_x     & F'_y     & 0 
  \end{array}\right| = 0
\end{equation}
\item un'altra cosa
\begin{eqnarray}
 y &=& x^4 + 4      \nonumber \\
   &=& (x^2+2)^2 -4x^2 \nonumber \\
   &\le&(x^2+2)^2
\end{eqnarray}
\item e un'ultima importantissima cosa
\begin{equation}
42
\end{equation}
\end{itemize}
\end{frame}

\begin{frame}
\frametitle{una slide con una figura}
\begin{figure}
\includegraphics[width=0.8\linewidth]{{figs/atmospheric_processes}.jpg}
\caption{Processi atmosferici rilevanti per la qualità dell'aria}
\end{figure}
\end{frame}



\end{document}
